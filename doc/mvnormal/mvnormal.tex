\documentclass{article}

\usepackage{amsmath}
\usepackage{amssymb}
\usepackage[osf]{mathpazo}
\usepackage{nicefrac}

\newcommand{\Mu}{M}
\newcommand{\R}{\mathbb{R}}
\newcommand{\T}{\intercal}
\newcommand{\code}[1]{\texttt{#1}}
\newcommand{\given}{\mathrel{|}}
\newcommand{\mapping}{\colon}
\newcommand{\normlx}[1]{#1\lVert}
\newcommand{\norml}{\left\lVert}
\newcommand{\normrx}[1]{#1\rVert}
\newcommand{\normr}{\right\rVert}
\newcommand{\normx}[2][]{\normlx{#1}#2\normrx{#1}}
\newcommand{\norm}[1]{\norml#1\normr}
\newcommand{\sfrac}[2]{\nicefrac{#1}{#2}}
\newcommand{\smallfrac}[2]{{\textstyle\frac{#1}{#2}}}

\newcommand{\n}[1]{\text{\rm #1}}
\newcommand{\zero}{\n0}
\newcommand{\one}{\n1}
\newcommand{\two}{\n2}

\DeclareMathOperator{\tr}{tr}

\newcommand{\fixincommand}[1]
{
  \let\elem\in
  \let\mele\ni

  \newcommand{\scriptify}[1]{\mathrel{\mathchoice
    {\raise.2ex\hbox{$\scriptstyle##1$}}
    {\raise.2ex\hbox{$\scriptstyle##1$}}
    {\raise.1ex\hbox{$\scriptscriptstyle##1$}}
    {\raise.1ex\hbox{$\scriptscriptstyle##1$}}}}

  \renewcommand{\in}{\scriptify{#1{\elem}}}
  \renewcommand{\ni}{\scriptify{#1{\elem}}}

  % We have to futz with a local resurrection of the original \in
  % command, because the built-in \notin command wants to use it.

  \let\nelem\notin

  \renewcommand{\notin}{\mathrel{\mathchoice
    {\raise.2ex\hbox{$\let\in\elem\scriptstyle#1{\nelem}$}}
    {\raise.2ex\hbox{$\let\in\elem\scriptstyle#1{\nelem}$}}
    {\let\in\elem\scriptscriptstyle#1{\nelem}}
    {\let\in\elem\scriptscriptstyle#1{\nelem}}}}
}
\fixincommand{\boldsymbol}

\begin{document}
\paragraph{Multivariate normal differential}
Let $Y$ be a multivariate normal variable with mean $\Mu$ and
 symmetric positive-definite covariance $\Sigma$, which has density
%
\begin{equation}
  \log P(Y \given \Mu, \Sigma)
    = -\smallfrac{\one}{\two}
        (Y - \Mu)^\T \Sigma^{-\one} (Y - \Mu)
      - \smallfrac{n}{\two} \log \two\pi
      - \smallfrac{\one}{\two} \log \det(\Sigma).
\end{equation}
%
Let $A = Y - \Mu$ and $\alpha = \Sigma^{-\one} A = \Sigma^{-\one} (Y -
 \Mu) = (A^\T \Sigma^{-\one})^\T$.
Then the differential of the density is
%
\begin{align}
  d \log P(Y \given \Mu, \Sigma)
   &= -\smallfrac{\one}{\two} \, d \bigl(
        A^\T \Sigma^{-\one} A
        + n \log \two\pi
        + \log \det(\Sigma)
      \bigr) \\
   &= -\smallfrac{\one}{\two} \bigl(
        d(A^\T \Sigma^{-\one} A)
        + d(n \log \two\pi)
        + d \log \det(\Sigma)
      \bigr).
\end{align}
%
Using the matrix calculus identities
%
\begin{align}
  d (U^\T) &= (dU)^\T, \\
  d (U V W) &= dU\,V W + U\,dV\,W + U V\,dW, \quad\text{and} \\
  d (U^{-\one}) &= -U^{-\one}\,dU\,U^{-\one};
\end{align}
%
 the vector inner/outer product identity
%
\begin{equation}
  u^\T v = \langle u, v \rangle = \tr(u \otimes v) = \tr(u v^\T);
\end{equation}
%
 and the fact that $\alpha^\T\,dA= (dA^\T\,\alpha)^\T = dA^\T\,\alpha$
 is a scalar, we have
%
\begin{align}
  d (A^\T \Sigma^{-\one} A)
   &= dA^\T\,\Sigma^{-\one} A
      + A^\T\,d(\Sigma^{-\one})\,A
      + A^\T \Sigma^{-\one}\,dA \\
   &= dA^\T\,\alpha
      - A^\T \Sigma^{-\one}\,d\Sigma\,\Sigma^{-\one} A
      + \alpha^\T\,dA \\
   &= \alpha^\T\,dA
      - \alpha^\T\,d\Sigma\,\alpha
      + \alpha^\T\,dA \\
   &= \two \alpha^\T\,dA
      - \tr(\alpha \alpha^\T \,d\Sigma).
\end{align}
%
Note that $d \log x = dx/x$ and $d \det(X) = \det(X) \tr(X^{-\one}\,dX)$, so
 that
%
\begin{equation}
  d \log \det(\Sigma)
    = \frac{d \det(\Sigma)}{\det(\Sigma)}
    = \frac{\det(\Sigma) \tr(\Sigma^{-\one}\,d\Sigma)}{\det(\Sigma)}
    = \tr(\Sigma^{-\one}\,d\Sigma).
\end{equation}
%
Thus, the density differential is
%
\begin{align}
  d \log P(Y \given \Mu, \Sigma)
   &= -\smallfrac{\one}{\two}
      \bigl[
        \two \alpha^\T\,d(Y - \Mu)
        - \tr(\alpha \alpha^\T\,d\Sigma)
        + \tr(\Sigma^{-\one}\,d\Sigma)
      \bigr] \\
   &= -\alpha^\T\,dY
      + \alpha^\T\,d\Mu
      - \smallfrac{\one}{\two}
        \tr\bigl((\alpha \alpha^\T - \Sigma^{-\one}) \,d\Sigma\bigr).
\end{align}

For a Gaussian process with mean function $m_\theta(X)$ and covariance
 kernel $k_\eta(X, X')$, the \code{make\_gp} SP's
 \code{gradientOfLogDensityOfData} method treats $X$ and $Y$ as fixed,
 and $\theta$ and $\eta$ as variable, with $\Mu = m_\theta(X)$ and
 $\Sigma = k_\eta(X, X)$, so that
%
\begin{equation}
  dY = \zero,
  \quad
  d\Mu = \frac{\partial\,m_\theta(X)}{\partial\theta}\,d\theta,
  \quad\text{and}\quad
  d\Sigma = \frac{\partial\,k_\eta(X, X)}{\partial\eta}\,d\eta;
\end{equation}
%
 then it computes the $d\theta$ and $d\eta$ components of the
 density differential, \textit{i.e.}~the gradient of $\log P(Y \given
 \Mu, \Sigma)$ with respect to the parameters $\theta$ and $\eta$ of
 the mean function and covariance kernel.

\paragraph{Conditional multivariate normal differential}
Let $Y = (Y_\one, Y_\two)$ be a multivariate normal variable with mean
 $\Mu = (\Mu_\one, \Mu_\two)$ and covariance
%
\begin{equation}
  \Sigma
    = \begin{pmatrix}
        \Sigma_{\one\one} & \Sigma_{\one\two} \\
        \Sigma_{\two\one} & \Sigma_{\two\two}
      \end{pmatrix}.
\end{equation}
%
The conditional distribution on $Y_\one$ given $Y_\two$ is another
 multivariate normal, with density
%
\begin{equation}
  \log P(Y_\one \given \Mu, \Sigma, Y_\two)
    = \log P(Y_\one \given \Mu_*, \Sigma_*),
\end{equation}
%
 where the conditional mean $\Mu_*$ and covariance $\Sigma_*$ are
\begin{align}
  \Mu_*
   &= \Mu_\one + \Sigma_{\one\two} \Sigma_{\two\two}^{-\one} (Y_\two - \Mu_\two), \\
  \Sigma_*
   &= \Sigma_{\one\one}
      - \Sigma_{\one\two} \Sigma_{\two\two}^{-\one} \Sigma_{\two\one}.
\end{align}
%
 and have differentials
%
\begin{align}
  d\Mu_*
   &= d\Mu_\one
      + d\Sigma_{\one\two}\,\Sigma_{\two\two}^{-\one} (Y_\two - \Mu_\two)
 \\&\quad\quad
      - \Sigma_{\one\two}
        \Sigma_{\two\two}^{-\one}\,d\Sigma_{\two\two}\,\Sigma_{\two\two}^{-\one}
        (Y_\two - \Mu_\two)
      + \Sigma_{\one\two} \Sigma_{\two\two}^{-\one}\,d(Y_\two - \Mu_\two),
    \nonumber\\
  d\Sigma_*
   &= d\Sigma_{\one\one}
      - d\Sigma_{\one\two}\,\Sigma_{\two\two}^{-\one} \Sigma_{\two\one}
 \\&\quad\quad
      + \Sigma_{\one\two}
        \Sigma_{\two\two}^{-\one}\,d\Sigma_{\two\two}\,\Sigma_{\two\two}^{-\one}
        \Sigma_{\two\one}
      - \Sigma_{\one\two} \Sigma_{\two\two}^{-\one}\,d\Sigma_{\two\one}.
    \nonumber
\end{align}

For a Gaussian process with mean function $m_\theta(X)$ and covariance
 kernel $k_\eta(X, X')$, with previously observed outputs $Y_\two$ for
 inputs $X_\two$ and unobserved outputs $Y_\one$ for inputs $X_\one$,
 the GP SP's \code{gradientOfLogDensity} method treats $\theta$,
 $\eta$, $X_\two$, and $Y_\two$ as fixed, and $X_\one$ and $Y_\one$ as
 variable.
In this case we have $dX_\two = dY_\two = d\Mu_\two =
 d\Sigma_{\two\two} = \zero$, and the remaining nonzero marginal mean
 and covariance differentials are
%
\begin{align}
  d\Mu_\one
   &= d\bigl(m_\theta(X_\one)\bigr) = m_\theta'(X_\one)\,dX_\one, \\
  d\Sigma_{\one\one}
   &= d\bigl(k_\eta(X_\one, X_\one)\bigr)
    = \partial_\one k_\eta(X_\one, X_\one)\,dX_\one
      + \partial_\two k_\eta(X_\one, X_\one)\,dX_\one \\
   &= \two\,\partial_\one k_\eta(X_\one, X_\one)\,dX_\one, \\
  d\Sigma_{\one\two}
   &= d\bigl(k_\eta(X_\one, X_\two)\bigr)
    = \partial_\one k_\eta(X_\one, X_\two)\,dX_\one,
      \quad\text{and} \\
  d\Sigma_{\two\one}
   &= d(\Sigma_{\one\two}^\T)
    = (d\Sigma_{\one\two})^\T.
\end{align}
%
From this, the conditional mean and covariance differentials reduce to
%
\begin{align}
  d\Mu_*
   &= d\Mu_\one
      + d\Sigma_{\one\two}\,\Sigma_{\two\two}^{-\one} (Y_\two - \Mu_\two) \\
   &= m_\theta'(X_\one)\,dX_\one
      + \partial_\one k_\eta(X_\one, X_\two)\,dX_\one
        \,\Sigma_{\two\two}^{-\one} (Y_\two - \Mu_\two), \\
  d\Sigma_*
   &= d\Sigma_{\one\one}
      - d\Sigma_{\one\two}\,\Sigma_{\two\two}^{-\one} \Sigma_{\two\one}
      - \Sigma_{\one\two} \Sigma_{\two\two}^{-\one}\,d\Sigma_{\two\one} \\
   &= d\Sigma_{\one\one}
      - \two\,d\Sigma_{\one\two}\,\Sigma_{\two\two}^{-\one} \Sigma_{\two\one} \\
   &= \two\,\partial_\one k_\eta(X_\one, X_\one)\,dX_\one
      - \two\,\partial_\one k_\eta(X_\one, X_\two)\,dX_\one
        \,\Sigma_{\two\two}^{-\one} \Sigma_{\two\one}.
\end{align}
%
Thus, if $\alpha = \Sigma_*^{-\one}(Y_\one - \Mu_*)$, the differential
 of the density is
%
\begin{align}
  d \log P&(Y_\one \given \Mu, \Sigma, Y_\two)
    = d \log P(Y_\one \given \Mu_*, \Sigma_*) \nonumber\\
   &= -\alpha^\T\,dY_\one
      + \alpha^\T \bigl[
          m_\theta'(X_\one)\,dX_\one \\
   &\quad\quad\quad\quad\quad\quad\quad\quad
          + \partial_\one k_\eta(X_\one, X_\two)\,dX_\one
            \,\Sigma_{\two\two}^{-\one} (Y_\two - \Mu_\two)
        \bigr] \nonumber\\
   &\quad\quad
      - \tr\bigl[
          (\alpha \alpha^\T - \Sigma_*^{-\one})
          \bigl(
            \partial_\one k_\eta(X_\one, X_\one)\,dX_\one \nonumber\\
   &\quad\quad\quad\quad\quad\quad\quad\quad\quad\quad\quad
            - \partial_\one k_\eta(X_\one, X_\two)\,dX_\one
              \,\Sigma_{\two\two}^{-\one} \Sigma_{\two\one}
          \bigr)
        \bigr]. \nonumber
\end{align}
%
The \code{gradientOfLogDensity} method computes the $dY_\one$ and
 $dX_\one$ components of this linear map from increments in $Y_\one$
 and $X_\one$ to increments in density.

Since the output space of the Gaussian process is $\R$, the array of
 output values is a vector in $\R^n$ and thus so is the $dY_\one$
 component of this differential.
If the input space of the Gaussian process is $\R$, the same goes for
 the $dX_\one$ component.
However, if the input space of the Gaussian process is $\R^k$ for some
 $k > \one$, the $dX_\one$ component of this differential is a linear
 map from an increment in a tuple $X_\one \in (\R^k)^n$ to an increment
 in a density in $\R$---and there is no numpy array representation of
 such a linear map.
Pooh.

\end{document}
